\documentclass[paper=a4]{scrartcl}

\usepackage[utf8]{inputenc}
\usepackage{algorithmic}
\usepackage{tcolorbox}
\usepackage{amsfonts}
\usepackage{amsmath}
\usepackage{amssymb}
\usepackage{textcomp}

\title{$3^{-n}$ Decimal Problem}
\author{Ting-Yo Kuo}
\date{}

\begin{document}
\maketitle
\setcounter{section}{0}
\section{Outline}
\subsection{Difinition of Problem}
Consider a problem:

\begin{tcolorbox}[arc=0pt,colback=white,title={Definition}]
For $n\in\mathbb{N}$, define 
\[
A_n=\frac{1}{3^n10^{3^k}}
\]
and
\[
A=\sum_{n=1}^\infty A_n
\]
\end{tcolorbox}
\begin{tcolorbox}[arc=0pt,colback=white,title={Definition}]
Given a real number $a=0.a_1a_2a_3\ldots\in[0,1)$ which does not end with repeating 9 and $n\in\mathbb{N}$, \textbf{the $n$-th term of $a$} is $a_n$.  
\end{tcolorbox}
Now given $n\in\mathbb{N}$,  find the $n$-th to $(n+9)$-th terms of $A$? Consider $n=10^2,10^8,10^{16}$.

\subsection{Outline of Solution}
Since $n$ is at most $10^{16}$, it's sufficient to approximate $A$ to be $\sum_{n=1}^{33}A_n$. Then we find $k$-th term of $A$ by finding around $k$-th terms of $A_n$ and sum them all. For $k\leq 16$, we find it in directly way, by finding its repetend. For $k>16$, let $a = 16, b = k-16$, then simulating $3^b$ divides the repetend of $1/3^a$. With some regularity, we can jump with the length $3^{a-2}$ when simulating. In conclusion, we get a solution with time complexity $O(\sqrt{n}\log n)$, which is good enough to halt in 1 minutes.

\section{Properties of the repetend of $1/3^k$}
\begin{tcolorbox}[arc=0pt,colback=white,title={Fact}]
\begin{itemize}
\item[1.] If $a\in\mathbb{Q}$, then $a$ has a repeating decimal representation.
\item[2.] If the repetend of $1/3^n=0.a_1\ldots a_{k}\overline{a_{k+1}\ldots a_{k+l}}$, then $k=0$.
\item[3.] \textbf{Long division} algorithm
\end{itemize}
\end{tcolorbox}
\begin{tcolorbox}[arc=0pt,colback=white,title={Definition}]
When $a\in[0,1)$ is in the form $0.a_1a_2\ldots a_{k}\overline{a_{k+1}a_{k+1}\ldots a_{k+l}}$ with least $k$ and $l$, and $a_{k+1},\ldots,a_{k+l}$ not all zero, $a_{k+1}\ldots a_{k+l}$ is called the \textbf{repetend} of $a$. The length of the repetend of $a$ is $l$.
\end{tcolorbox}
\begin{tcolorbox}[arc=0pt,colback=white,title={Property}]
The length of the repetend of $1/3$ is 1. For $n\geq 2$, the length of the repetend of $1/3^n$ is $3^{n-2}$.
\tcblower
\textit{Proof. } Induction on $n$. $1/3=0.\overline 3$ and $1/9=0.\overline 1$ are trivial. For $1/3^n$, we generate the repetend of $1/3^n$ by  3 dividing the repetend of $1/3^{n-1}$.  Let the repetend of $1/3^{n-1}=0.\overline{a_1\ldots a_{3l}}$ and $1/3^{n-2}=0.\overline{b_1\ldots b_l}$. If $a_1\ldots a_{3l}=0\pmod{3}$, then $b_1\ldots b_lb_1\ldots b_lb_1\ldots b_l=0\pmod{9}$, thus $b_1\ldots b_l=0\pmod 3$. Then the length of the repetend of $1/3^{n-1}$ can be $l<3l$, a contradiction. If $a_0\ldots a_{3l}=1\pmod 3$, then $a_1\ldots a_{3l}a_1\ldots a_{3l}=10\ldots 01=2\pmod 3$. And $a_1\ldots a_{3l}a_1\ldots a_{3l}a_1\ldots a_{3l}=10\ldots 010\ldots 01=3=0\pmod 3$. Hence the length of the repetend of $1/3^n$ is a factor of $9l$. On the other hand, it is similar for the case $a_1\ldots a_{3l}=2\pmod 3$. Also, it is larger than $3l$, therefore, it is exactly $9l$. By induction, $9l=3^{n-2}$.
\end{tcolorbox}

\begin{tcolorbox}[arc=0pt,colback=white,title={Property}]
Let $n,m\in\mathbb{N}\cup\{0\}$, then 
\[
10^{n3^{m-2}}=1\pmod{3^m}
\]
\tcblower
\textit{Proof.} Consider $n=1$. Induction on $m$. Clear for $m=3,9$. Given $m$, let $n=m-2$. By induction, $10^{3^{n-3}}=1\pmod{3^{m-1}}$, then $10^{3^{n-2}}=10^{3^{n-3}}10^{3^{n-3}}10^{3^{n-3}}$. Thus $10^{3^{n-2}}=1\pmod{3^m}$. And it's clear for $n=0$ and $n>1$.
\end{tcolorbox}

\section{Algorithm to solve the problem}
We consider solving $A_n$ only, then we can get solution by summing $A_1$ to $A_{33}$.
\begin{tcolorbox}[arc=0pt,colback=white,title={Algorithm}]
Find the repetend of $3^{-n}$, brutal method.
\tcblower
\begin{algorithmic}
\STATE Let $V:=\emptyset,d:=1,A=()$
\WHILE {$d\not\in V$} 
	\STATE $V:=V\cup\{d\}$
    \STATE $d:=10d$
    \STATE Let $a\in\mathbb{N}\cup\{0\}$ be maximum such that $a3^n\leq d$
    \STATE $A:=(A,a)$
    \STATE $d:=d-a3^n$
\ENDWHILE 
\RETURN $A$
\end{algorithmic}
\end{tcolorbox}

The above algorithm is allowed for $n\leq 17$, at the limit of time and space.

\section{Derivation for $n>17$}

$\ \ $For $n>17$, let $m=17,l=n-17$. By long division algorithm, to get the $k$-th term of $1/3^{n}$, we can simulate $\frac{3^{-m}}{3^l}$, where we replace 
$3^{-m}$ with $0.\overline{a_1\ldots a_{3^{m-2}}}$, the repetend of $3^{-m}$. To get the $k$-th term of $3^{-n}$, it's equivalent to first compute the remainder, $a_1\ldots a_{k-1}\mod 3^l$, where we define $a_i=a_j\Leftrightarrow i=j\pmod{3^{m-2}}$. Next, we split $a_1\ldots a_{k-1}$ into $N$ or $N+1$ pieces, where $N$ is maximum satisfying $N3^{m-2}\leq k-1$, say $a_1\ldots a_{N3^{m-2}}a_{N3^{m-2}+1}\ldots a_{k-1}$. Although $N\approx 10^{16}$, we have fast algorithm to compute $a_1\ldots a_{N3^{m-2}}\mod 3^l$, by

\begin{equation}
\begin{split}
  & a_1\ldots a_{N3^{m-2}} \pmod{3^l}\\
= & (a_1\ldots a_{3^{m-2}})(10^{(N-1)3^{m-2}}+10^{(N-2)3^{M-2}}+\ldots +10^0)\pmod{3^l}\\
= & (a_1\ldots a_{3^{m-2}}\mod 3^l)\sum_{i=0}^{N-1}(10^{i3^{m-2}}\mod 3^l)\\
= & (a_1\ldots a_{3^{m-2}}\mod 3^l)N
\end{split}
\end{equation}

Hence we have the following method.
\begin{tcolorbox}[arc=0pt,colback=white,title={Algorithm}]
Given $k,n\in\mathbb{N}$, $n>17$, find $k$-th term of $3^{-n}$
\tcblower
\begin{algorithmic}
\STATE $m:=17$, $l:=n-17$, $a_1\ldots a_{3^{m-2}}:=$ the repetend of $3^{-m}$
\STATE $N:=$ the maximum satisfying $N3^{m-2}\leq k-1$
\STATE $a:=(a_1\ldots a_{3^{m-2}}\mod 3^l)N\mod 3^l$
\RETURN $aa_1a_2\ldots a_{k-1-N3^{m-2}}\mod 3^l$
\end{algorithmic}
\end{tcolorbox}
Thus we have a method to get $k$-th term of $A_n$. Each step of the above algorithm is $O(\sqrt k)$. Thus totally $O(\sqrt k\log k)$ to find all $A_n$, $1\leq n\leq 33$.

\section{Carry of the summation of $A_n$}
The problem ask $(n-1)$-th term to $(n+9)$-th terms, but we can't only fetch those term, we also need to take $(n+10)$-th, $(n+11)$-th or more into consideration, since it may carry to $(n+9)$-th term of $A$.
\begin{tcolorbox}[arc=0pt,colback=white,title={Property}]
For $n=10^{16}$, we need to consider at most $(n+11)$-th term.
\tcblower \textit{Proof.} Let the carry from $3^{-34}$-th term be $c$. Then the carry from $(3^{-34}-1)$-th term is at most $\lfloor (c+9\times 33)/10\rfloor$, the carry from $(3^{-34}-2)$-th term is at most $\lfloor(\lfloor(c+9\times 33)/10\rfloor+9\times 33)/10\rfloor$. Define a sequence $a_1=c$, $a_n=\lfloor(a_{n-1}+297)/10\rfloor$. Since $a_n$ is bounded above by sequence $b_1=c$, $b_n=(b_{n-1}+297)/10$. And solve it that $b_n=b_1\times 10^{-(n-1)}+29.7\sum_{i=0}^{n-2}10^{i}=c10^{1-n}+29.7/90(10^n-10)$. Now take $n=3^{34}-3^{33}+9$, which is large enough that $a_n$ is at most 2 digits. Thus if we consider $(n+11)$-th term, it will not affect $(n+9)$-th term.
\end{tcolorbox}

\section{Conclude}
In conclusion, we have the following method to solve the problem.
\begin{tcolorbox}[arc=0pt,colback=white,title={Algorithm}]
Given $k\leq 10^{16}$, find the $(k-1)$-th term to $(k+9)$-th term of $A$.
\tcblower
\begin{algorithmic}
\FOR {$n=1$ \TO $33$}
	\IF {$k\leq 3^{17}$}
		\STATE apply algorithm of section 3, find $(k-1)$-th to $(k+11)$-th term of $A_n$
	\ELSE 
		\STATE apply algorithm of section 4, find $(k-1)$-th to $(k+11)$0th term of $A_n$.
	\ENDIF
\ENDFOR
\STATE add those terms
\RETURN the $(k-1)$-th to $(k+9)$-th term
\end{algorithmic}
\end{tcolorbox}
\end{document}